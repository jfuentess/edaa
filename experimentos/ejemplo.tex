\documentclass{scrartcl}

\usepackage[T1]{fontenc}
\usepackage[utf8]{inputenc}
\usepackage[spanish]{babel}

\usepackage{mathtools}
\usepackage{amssymb}
\usepackage{amsthm}

\usepackage{lmodern}
\usepackage{microtype}

\usepackage{pgfplots}
\pgfplotsset{compat=1.18}

\begin{document}
\title{Tablas y gráficos}
\author{Leonardo Lovera}

\maketitle

A continuación mostramos algunos ejemplos usuales de gráficos generados con los \texttt{.csv}
generados por \texttt{uhr}. Luego, veremos las tablas correspondientes generadas por \texttt{csvltx}.

\section*{Gráficos}

\begin{center}
    \begin{tikzpicture}[scale=0.8]
        \begin{loglogaxis}[title={log-log}, log base x = 2, xtick=data, xlabel={x axis label}, ylabel={y axis label}]
            \addplot table [x=n, y=t_mean, col sep=comma] {experimental_data/log-log_ms.csv};
        \end{loglogaxis}
    \end{tikzpicture}
\end{center}

\begin{center}
    \begin{tikzpicture}[scale=0.8]
        \begin{axis}[title={Super cool and descriptive name}, xlabel={Unit length variable}, ylabel={Response}]
            \addplot table [x=p, y=t_mean, col sep=comma] {experimental_data/cool_ms.csv};
        \end{axis}
    \end{tikzpicture}
\end{center}

\begin{center}
    \begin{tikzpicture}[scale=0.8]
        \begin{semilogxaxis}[title={semi-log plot}, log base x = 2, xlabel={Position}, ylabel={Average happiness}, ymin=275, ymax=285]
            \addplot table [x=p, y=t_mean, col sep=comma] {experimental_data/semi-log_ns.csv};
        \end{semilogxaxis}
    \end{tikzpicture}
\end{center}

Si solamente se quiere generar un gráfico y no necesariamente usar \LaTeX, sigue sirviendo esta
técnica cambiando el \texttt{documentclass} a \texttt{standalone}, lo que permite generar gráficos en
\texttt{.pdf} portables para otros propósitos.

\section*{Tablas}

\begin{center}
\begin{tabular}{|c|c|c|}
    \hline
    \multicolumn{3}{|c|}{\texttt{log-log\_ms}} \\
    \hline
    \texttt{n} & \texttt{t\_mean} & \texttt{t\_stdev} \\
    \hline
    8 & 5.30802e-05 & 1.14243e-06 \\
    \hline
    64 & 0.000160142 & 3.68659e-06 \\
    \hline
    512 & 0.000931126 & 4.00216e-05 \\
    \hline
    4096 & 0.00719646 & 0.000932411 \\
    \hline
    32768 & 0.0610976 & 0.00589483 \\
    \hline
    262144 & 0.485865 & 0.0170079 \\
    \hline
    2097152 & 4.02294 & 0.0962575 \\
    \hline
    16777216 & 32.1973 & 1.40627 \\
    \hline
    134217728 & 262.37 & 18.0376 \\
    \hline
\end{tabular}

\begin{tabular}{|c|c|c|}
    \hline
    \multicolumn{3}{|c|}{\texttt{cool\_ms}} \\
    \hline
    \texttt{p} & \texttt{t\_mean} & \texttt{t\_stdev} \\
    \hline
    0 & 4.5075e-05 & 1.75632e-06 \\
    \hline
    0.125 & 4.04943 & 0.138594 \\
    \hline
    0.25 & 8.05313 & 0.0954546 \\
    \hline
    0.375 & 12.0908 & 0.0962799 \\
    \hline
    0.5 & 16.1394 & 0.176301 \\
    \hline
    0.625 & 20.2035 & 0.47803 \\
    \hline
    0.75 & 24.191 & 1.4683 \\
    \hline
    0.875 & 28.2644 & 0.26736 \\
    \hline
    1 & 32.3056 & 0.30478 \\
    \hline
\end{tabular}

\begin{tabular}{|c|c|c|}
    \hline
    \multicolumn{3}{|c|}{\texttt{semi-log\_ns}} \\
    \hline
    \texttt{p} & \texttt{t\_mean} & \texttt{t\_stdev} \\
    \hline
    0 & 288.835 & 2.7205 \\
    \hline
    2097151 & 279.319 & 2.23414 \\
    \hline
    4194302 & 280.32 & 2.07906 \\
    \hline
    6291453 & 279.507 & 2.08915 \\
    \hline
    8388604 & 279.166 & 1.87087 \\
    \hline
    10485755 & 279.739 & 2.2409 \\
    \hline
    12582906 & 279.489 & 2.00447 \\
    \hline
    14680057 & 279.358 & 2.17522 \\
    \hline
    16777208 & 278.802 & 2.27606 \\
    \hline
\end{tabular}
\end{center}

Todas estas tablas fueron generadas usando \texttt{csvltx} y en las últimas dos se modificó el heading
para que dijeran \texttt{p} en vez de \texttt{n}. 
\end{document}